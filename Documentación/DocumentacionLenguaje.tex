\documentclass[10pt,a4paper]{report}
\usepackage[latin1]{inputenc}
\usepackage{amsmath}
\usepackage{amsfonts}
\usepackage{amssymb}
\begin{document}
 
Documentaci\'on Lenguaje: \\\\
1) Celdas: $A1$, $B4$, $K2$, $L11$ \\\\
2) N\'umeros: $2$, $1$, $9$, $3.14$, $2.1823$, $-9$, $-9.22$\\\\
3) Booleanos: $true$, $false$ \\\\
4) Strings: $"hola\ mundo"$, $"esta\ es\ una\ string"$ \\\\
5) Rango: $Celda1:Celda2$ (regi\'on rectangular de la celda 1 a la celda 2) \\\\
6) D\'ias: $dd/mm/yyyy$ \\\\
7) Operaciones Matem\'aticas: $+$ (suma), $-$ (resta), $/$ (division), $*$ (multiplicacion), $\wedge$ (potencia) \\\\
8) Operaciones Booleanas: $\&$ (and), $|$ (or) \\\\
9) Comparaciones Matem\'aticas: $<$ (menor), $<=$ (menor o igual), $>$ (mayor), $>=$ (mayor o igual), $==$ (igual) \\\\
10) Funciones Matem\'aticas: \\
 $suma(a,b,c,d,...)$ (sumatoria de n\'umeros) \\
 $abs(a)$ (valor absoluto de un n\'umero) \\\\
11) Funciones de Strings: \\
  $concat(a,b,c,d,...)$ (concatenaci\'on de todas las strings pasadas) \\\\
12) Funciones de Fechas: \\
  $hoy()$ (la fecha de hoy) \\
  $diasEntre(fecha_1,fecha_2)$ (diferencia entre dos d\'ias) \\
  $diaPascua(year)$ (calcula el d\'ia de pascua dado un anio) \\\\
13) Sentencias:\\ $si(Expresion\ Booleana,Expresion1,Expresion2)$ (sentencia si), \\ $contarSi(booleana(!),Rango)$ (cuenta las celdas del rango que cumplen con una funci\'on booleana definida con variable $!$) \\\\
\end{document}